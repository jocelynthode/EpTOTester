\section{Conclusion}
\label{sec:conclusion}
Benchmarking large scale distributed systems is a hard task. One needs to have access to a rather large cluster, possibly on different sites to reproduce real world latencies. These benchmarks can cost a lot forcing researchers to rely on simulations. Unfortunately, simulations often disregard small but important details that only happen in a real world scenario.

In this paper, we implement \eptotester an architecture designed to facilitate the deployment of a large-scale distributed system on a cluster and the injection of churn in the network. Furthermore, we implement \epto and evaluate the claims made in \autocite{matos2015epto} against \jgroups SEQUENCER, a total order algorithm relying on a coordinator. Unfortunately, our cluster cannot handle more than 100 nodes at the same time. Due to this limitation, \epto performs worse than \jgroups. We want to reiterate that this is expected as \epto is designed for large distributed networks. \jt{Let's write it this way for the moment if we are able to run something on GCE or AWS we can always change it later.}