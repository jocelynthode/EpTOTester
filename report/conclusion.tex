\section{Conclusion}
\label{sec:conclusion}
Benchmarking large scale distributed systems is a hard task. One needs to have access to a rather large cluster, possibly on different sites to reproduce real world latencies. These benchmarks can cost a lot forcing researchers to rely on simulations. Unfortunately, simulations often disregard small but important details that only happen in a real world scenario.

In this paper, we implement \sys a framework designed to facilitate the deployment of a large-scale distributed system on a cluster and the injection of churn in the network. Furthermore, we implement \epto and evaluate the claims made in \autocite{matos2015epto} against \jgroups SEQUENCER, a total order algorithm relying on a coordinator. Unfortunately, our cluster cannot handle more than 100 nodes at the same time. Due to this limitation, \epto performs worse than \jgroups.
Still, by extrapolating the observed behavior of \epto and \jgroups to a larger number of peers than we tested we can clearly see that \jgroups, albeit being more efficient, scales worse than \epto.
Further experiments, in larger settings are needed to confirm this.
