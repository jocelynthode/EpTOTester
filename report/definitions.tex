\section{Definitions}
\label{sec:definitions}
\mm{several of these terms appear before, this needs to be moved upwards.}
A process or peer is defined as an actor in our system running the application that needs total order. Each process communicates with other processes in the distributed system, exchange events, and order them together.
\par
An event $j$ is defined as: $$e_j \triangleq (broadcasterID,timestamp,payload)$$ where the timestamp is the local time at the process when the event is created and is used with the broadcasterID to order events. The payload represents the data sent.
\par
A ball is an abstraction taken from the balls-and-bins problem. In practice, we bundle sets of events together and send them as one packet to reduce the network traffic overhead.

The number of rounds needed to deliver an event, the fan-out and the number of messages transmitted per process per round are all logarithmic in the number of processes. \autocite{matos2015epto}
\par
Since \epto uses a probabilistic agreement instead of a deterministic agreement, there is a nonzero probability that a peer does not receive an event. In this instance there will be a hole in the sequence of delivered events, but the order of delivered events is protected by \epto's deterministic ordering algorithm, thus the total order property is preserved. The probability that a peer does not receive an event is controlled by \epto and depends on various parameters such as the fan-out, the number of rounds (time to live), the number of nodes, and the network conditions. In particular, this probability can be made smaller than the probability of catastrophic network failure. %\hm{I rewrote this paragraph. Please reread.}\jt{Done.}
