
\section{Evaluation}
\label{sec:evaluation}
\jt{specify somewhere the problems encountered with JGroups}
\pgfplotstableread[col sep=comma]{csv-data/average-bandwidth.csv}\tableaverage
\pgfplotstableread[col sep=comma]{csv-data/average-bandwidth-real-trace.csv}\tableaveragereal
\pgfplotstableread[col sep=comma]{csv-data/plot-global-time-cdf.csv}\tableglobaltime
\pgfplotstableread[col sep=comma]{csv-data/plot-local-deltas-cdf.csv}\tablelocaldeltas
\pgfplotstableread[col sep=comma]{csv-data/plot-local-time-cdf.csv}\tablelocaltime
In this section we present our results. In the following benchmarks \epto uses a constant $c$ to set the probability of a hole appearing. We set $c = 4$ in all our benchmarks so as to not experience any holes as \jgroups does not produce holes under normal conditions. We use a $\delta$ period of \SI{100}{\milli\second} for tests with no churn or synthethic churn and \SI{250}{\milli\second} for tests following a real trace.
\par 
We used specific settings recommended by \jgroups for a big cluster.
\par
Every \jgroups test run with churn is run once killing the coordinator and once not killing it. We use the following nomenclature to differentiate both tests:
\begin{description}[\IEEEsetlabelwidth{\jgroups-nocoord: }]
	\item[\textbf{\jgroups-coord}:] The coordinator is killed once.
	\item[\textbf{\jgroups-nocoord}:] The coordinator is purposely kept alive throughout the experiment.
\end{description}
\par
\textbf{Testbed.} \jt{Most of this paragraph is copied from ErasureBench as we use the same cluster and Docker as well. Is this fine as is?}Our cluster consists of a cluster of 20 machines interconnected by a \SI{1}{\giga\bit\per\second} switched network. Each machine has an 8-core Intel Xeon CPU and \SI{8}{\giga\byte} of RAM. We deploy 12 virtual machines on these hosts. Each virtual machine has acces to 4 CPUs and 4 VCPUs as well as \SI{7}{\giga\byte} of RAM. We use KVM as our hypervisor. Each VM uses Debian as its O/S. EpTOTester is packaged as docker images. We use Docker 1.12 and its new functionality Docker Swarm to orchestrate our services. Each docker container has a memory restriction of \SI{300}{\mega\byte} of RAM to be able to run them all together.
\par
\textbf{Experiment parameters.} Every benchmark except the one following a real trace are run 10 times, each during \SI{20}{\minute}. When there is synthetic churn, the churn starts \SI{30}{\second} after the benchmark and run for \SI{17}{\minute}. The benchmarks following a real trace are run 5 times. The trace is speed-up 2x, which means we follow \SI{2}{\hour} worth of trace in \SI{1}{\hour}. Every benchmark run with churn is run with the $(100,50)$ parameters. When we have a global even throughput lower than the number of actual peers, we use a uniform random number generator to artificially reduce the global event throughput. For example, using $(100,50)$ parameters, each peer has a 50\% probability of sending an event at each second.
\par
We compared \epto and \jgroups based on their bandwidth, the time to deliver every expected events, their local dissemination stretch and finally the number of events sent.
\subsection{Bandwidth}
The initial bandwidth peaks observed for \epto are due to the PSS initialization. When a node starts, it received an initial view and then quickly run the PSS four times to obtain a random enough view.
\begin{figure}[hpt]
	\centering
	\begin{tikzpicture}
\usetikzlibrary{plotmarks}
\pgfplotsset{
	height=4cm,
	width=\linewidth / 2.6,
	every axis plot post/.append style={
		solid,
		very thin,
		mark=none
	},
	/pgfplots/area cycle list/.style={/pgfplots/cycle list={%
			{black,fill=black,mark=none},%
			{black,fill=white!25!black,mark=none},%
			{black,fill=white!50!black,mark=none},%
			{black,fill=white!75!black,mark=none},%
			{black,fill=white,mark=none},%
		}
	},
}
\begin{groupplot}[
ymajorgrids,
group style={
	group size=3 by 2,
	vertical sep=8mm,
	horizontal sep=4mm,
	xlabels at=edge bottom,
	ylabels at=edge left,
	yticklabels at=edge left,
},
stack plots=y,area style, enlarge x limits=false, 
ymin=0,
xmin=0,
ymax=4.5,
ytick={0,1,2,3,4},
yticklabels={0,1,2,3,4},
ylabel={Bandwidth $\left[\SI{}{\mbps}\right]$},
xlabel={Time $\left[\si{\minute}\right]$},
legend columns=5,
legend cell align=left,
legend style={at={(1.8,1.5)},anchor=north, font=\small, draw=none, fill=none},]
\nextgroupplot[ymax=4,ytick={0,1,2,3,4},
yticklabels={0,1,2,3,4},]
\addplot table[x=time,y=EpTO-50-1sec-0.000000, col sep=comma]{\tableaverage} \closedcycle;
\addplot table[x=time,y=EpTO-50-1sec-0.250000, col sep=comma]{\tableaverage} \closedcycle;
\addplot table[x=time,y=EpTO-50-1sec-0.500000, col sep=comma]{\tableaverage} \closedcycle;
\addplot table[x=time,y=EpTO-50-1sec-0.750000, col sep=comma]{\tableaverage} \closedcycle;
\addplot table[x=time,y=EpTO-50-1sec-1.000000, col sep=comma]{\tableaverage} \closedcycle;
\legend{0, 0.25, 0.5, 0.75, 1}
%
\nextgroupplot[ymax=4,ytick={0,1,2,3,4},]
\addplot table[x=time,y=EpTO-50-2sec-0.000000, col sep=comma]{\tableaverage} \closedcycle;
\addplot table[x=time,y=EpTO-50-2sec-0.250000, col sep=comma]{\tableaverage} \closedcycle;
\addplot table[x=time,y=EpTO-50-2sec-0.500000, col sep=comma]{\tableaverage} \closedcycle;
\addplot table[x=time,y=EpTO-50-2sec-0.750000, col sep=comma]{\tableaverage} \closedcycle;
\addplot table[x=time,y=EpTO-50-2sec-1.000000, col sep=comma]{\tableaverage} \closedcycle;
%
\nextgroupplot[ymax=4,ytick={0,1,2,3,4},]
\addplot table[x=time,y=EpTO-100-1sec-0.000000, col sep=comma]{\tableaverage} \closedcycle;
\addplot table[x=time,y=EpTO-100-1sec-0.250000, col sep=comma]{\tableaverage} \closedcycle;
\addplot table[x=time,y=EpTO-100-1sec-0.500000, col sep=comma]{\tableaverage} \closedcycle;
\addplot table[x=time,y=EpTO-100-1sec-0.750000, col sep=comma]{\tableaverage} \closedcycle;
\addplot table[x=time,y=EpTO-100-1sec-1.000000, col sep=comma]{\tableaverage} \closedcycle;
%
\nextgroupplot[ymax=1.5,ytick={0,0.5,1,1.5},yticklabels={0,0.5,1,1.5},]
\addplot table[x=time,y=JGroups-50-1sec-0.000000, col sep=comma]{\tableaverage} \closedcycle;
\addplot table[x=time,y=JGroups-50-1sec-0.250000, col sep=comma]{\tableaverage} \closedcycle;
\addplot table[x=time,y=JGroups-50-1sec-0.500000, col sep=comma]{\tableaverage} \closedcycle;
\addplot table[x=time,y=JGroups-50-1sec-0.750000, col sep=comma]{\tableaverage} \closedcycle;
\addplot table[x=time,y=JGroups-50-1sec-1.000000, col sep=comma]{\tableaverage} \closedcycle;
%
\nextgroupplot[ymax=1.5,ytick={0,0.5,1,1.5},]
\addplot table[x=time,y=JGroups-50-2sec-0.000000, col sep=comma]{\tableaverage} \closedcycle;
\addplot table[x=time,y=JGroups-50-2sec-0.250000, col sep=comma]{\tableaverage} \closedcycle;
\addplot table[x=time,y=JGroups-50-2sec-0.500000, col sep=comma]{\tableaverage} \closedcycle;
\addplot table[x=time,y=JGroups-50-2sec-0.750000, col sep=comma]{\tableaverage} \closedcycle;
\addplot table[x=time,y=JGroups-50-2sec-1.000000, col sep=comma]{\tableaverage} \closedcycle;
%
\nextgroupplot[ymax=1.5,ytick={0,0.5,1,1.5},]
\addplot table[x=time,y=JGroups-100-1sec-0.000000, col sep=comma]{\tableaverage} \closedcycle;
\addplot table[x=time,y=JGroups-100-1sec-0.250000, col sep=comma]{\tableaverage} \closedcycle;
\addplot table[x=time,y=JGroups-100-1sec-0.500000, col sep=comma]{\tableaverage} \closedcycle;
\addplot table[x=time,y=JGroups-100-1sec-0.750000, col sep=comma]{\tableaverage} \closedcycle;
\addplot table[x=time,y=JGroups-100-1sec-1.000000, col sep=comma]{\tableaverage} \closedcycle;
\end{groupplot}
%
\node[anchor=south] at (group c1r1.north) {$(50,50)$};
\node[anchor=south] at (group c2r1.north) {$(50,100)$};
\node[anchor=south] at (group c3r1.north) {$(100,50)$};
\node[anchor=south, rotate=-90] at (group c3r1.east){\epto};
\node[anchor=south, rotate=-90] at (group c3r2.east){\jgroups};
\end{tikzpicture}
	\vspace{-2mm} 
	\caption{Bandwidth percentiles of a node during an experiment}
	\vspace{-2mm} 
	\label{fig:bandwidth}
\end{figure}
In \autoref{fig:bandwidth} we observe that \epto has a worse baseline compared to \jgroups. It uses a median bandwidth of approximately \SI{1}{\mbps} for $(50,50)$ whereas \jgroups uses a median bandwidth of less than \SI{0.2}{\mbps}. However, in \jgroups most of the work is done solely by the coordinator. We can clearly see this as the 100th percentile is much higher than the rest and uses approximately \SI{.6}{\mbps}.

Comparing \epto and \jgroups in terms of bandwidth when we increase the number of events sent per second, we can see the bandwidth doubling in both cases. In lower peers scenario such as the ones presented in \autoref{fig:bandwidth} \jgroups is clearly at an advantage. Since \epto has a worse baseline we will reach the maximum bandwidth possible much quicker when increasing the event throughput.

Comparing \epto and \jgroups in terms of bandwidth when we increase the number of peers, \epto scales better than \jgroups. Where \jgroups basically has to double the bandwidth usage of the coordinator, \epto only increases it marginally \jt{logarithmic I think as was expected}. Thus in a scenario where we have many peers \epto will be more efficient than \jgroups at not reaching the maximum bandwidth.

\begin{figure}[hpt]
	\centering
	\begin{tikzpicture}
\usetikzlibrary{plotmarks}
\pgfplotsset{
	height=4cm,
	width=\linewidth / 2,
	every axis plot post/.append style={
		solid,
		very thin,
		mark=none
	},
	/pgfplots/area cycle list/.style={/pgfplots/cycle list={%
			{black,fill=black,mark=none},%
			{black,fill=white!25!black,mark=none},%
			{black,fill=white!50!black,mark=none},%
			{black,fill=white!75!black,mark=none},%
			{black,fill=white,mark=none},%
		}
	},
}
\begin{groupplot}[
ymajorgrids,
group style={
	group size=2 by 3,
	vertical sep=8mm,
	horizontal sep=4mm,
	xlabels at=edge bottom,
	ylabels at=edge left,
	yticklabels at=edge left,
},
stack plots=y,area style, enlarge x limits=false, 
ymin=0,
xmin=0,
ylabel={Bandwidth $\left[\SI{}{\mbps}\right]$},
xlabel={Time $\left[\si{\minute}\right]$},
legend columns=5,
legend cell align=left,
legend style={at={(1.12,1.5)},anchor=north, font=\small, draw=none, fill=none},]
\nextgroupplot[ymax=4,ytick={0,1,2,3,4},
yticklabels={0,1,2,3,4},]
\addplot table[x=time,y=EpTO-suspend-0.000000, col sep=comma]{\tableaverage} \closedcycle;
\addplot table[x=time,y=EpTO-suspend-0.250000, col sep=comma]{\tableaverage} \closedcycle;
\addplot table[x=time,y=EpTO-suspend-0.500000, col sep=comma]{\tableaverage} \closedcycle;
\addplot table[x=time,y=EpTO-suspend-0.750000, col sep=comma]{\tableaverage} \closedcycle;
\addplot table[x=time,y=EpTO-suspend-1.000000, col sep=comma]{\tableaverage} \closedcycle;
\legend{0, 0.25, 0.5, 0.75, 1}
%
\nextgroupplot[ymax=4,ytick={0,1,2,3,4},]
\addplot table[x=time,y=EpTO-add-suspend-0.000000, col sep=comma]{\tableaverage} \closedcycle;
\addplot table[x=time,y=EpTO-add-suspend-0.250000, col sep=comma]{\tableaverage} \closedcycle;
\addplot table[x=time,y=EpTO-add-suspend-0.500000, col sep=comma]{\tableaverage} \closedcycle;
\addplot table[x=time,y=EpTO-add-suspend-0.750000, col sep=comma]{\tableaverage} \closedcycle;
\addplot table[x=time,y=EpTO-add-suspend-1.000000, col sep=comma]{\tableaverage} \closedcycle;
%
\nextgroupplot[ymax=20,ytick={0,5,10,15,20},
yticklabels={0,5,10,15,20},]
\addplot table[x=time,y=JGroups-suspend-coord-0.000000, col sep=comma]{\tableaverage} \closedcycle;
\addplot table[x=time,y=JGroups-suspend-coord-0.250000, col sep=comma]{\tableaverage} \closedcycle;
\addplot table[x=time,y=JGroups-suspend-coord-0.500000, col sep=comma]{\tableaverage} \closedcycle;
\addplot table[x=time,y=JGroups-suspend-coord-0.750000, col sep=comma]{\tableaverage} \closedcycle;
\addplot table[x=time,y=JGroups-suspend-coord-1.000000, col sep=comma]{\tableaverage} \closedcycle;
%
\nextgroupplot[ymax=20,ytick={0,5,10,15,20},]
\addplot table[x=time,y=JGroups-add-suspend-coord-0.000000, col sep=comma]{\tableaverage} \closedcycle;
\addplot table[x=time,y=JGroups-add-suspend-coord-0.250000, col sep=comma]{\tableaverage} \closedcycle;
\addplot table[x=time,y=JGroups-add-suspend-coord-0.500000, col sep=comma]{\tableaverage} \closedcycle;
\addplot table[x=time,y=JGroups-add-suspend-coord-0.750000, col sep=comma]{\tableaverage} \closedcycle;
\addplot table[x=time,y=JGroups-add-suspend-coord-1.000000, col sep=comma]{\tableaverage} \closedcycle;
%
\nextgroupplot[ymax=2,ytick={0,0.5,1,1.5,2},
yticklabels={0,0.5,1,1.5,2},]
\addplot table[x=time,y=JGroups-suspend-nocoord-0.000000, col sep=comma]{\tableaverage} \closedcycle;
\addplot table[x=time,y=JGroups-suspend-nocoord-0.250000, col sep=comma]{\tableaverage} \closedcycle;
\addplot table[x=time,y=JGroups-suspend-nocoord-0.500000, col sep=comma]{\tableaverage} \closedcycle;
\addplot table[x=time,y=JGroups-suspend-nocoord-0.750000, col sep=comma]{\tableaverage} \closedcycle;
\addplot table[x=time,y=JGroups-suspend-nocoord-1.000000, col sep=comma]{\tableaverage} \closedcycle;
%
\nextgroupplot[ymax=2,ytick={0,0.5,1,1.5,2},]
\addplot table[x=time,y=JGroups-add-suspend-nocoord-0.000000, col sep=comma]{\tableaverage} \closedcycle;
\addplot table[x=time,y=JGroups-add-suspend-nocoord-0.250000, col sep=comma]{\tableaverage} \closedcycle;
\addplot table[x=time,y=JGroups-add-suspend-nocoord-0.500000, col sep=comma]{\tableaverage} \closedcycle;
\addplot table[x=time,y=JGroups-add-suspend-nocoord-0.750000, col sep=comma]{\tableaverage} \closedcycle;
\addplot table[x=time,y=JGroups-add-suspend-nocoord-1.000000, col sep=comma]{\tableaverage} \closedcycle;
%
\end{groupplot}
\node[anchor=south] at (group c1r1.north) {1 kill};
\node[anchor=south] at (group c2r1.north) {1 \{kill,add\}/minute};
\node[anchor=south, rotate=-90] at (group c2r1.east){\epto};
\node[anchor=south, rotate=-90] at (group c2r2.east){\jgroups-coord};
\node[anchor=south, rotate=-90] at (group c2r3.east){\jgroups-nocoord};
\end{tikzpicture}
	\vspace{-2mm} 
	\caption{Bandwidth percentiles of a node during an experiment with churn}
	\vspace{-2mm} 
	\label{fig:bandwidth-churn}
\end{figure}

In \autoref{fig:bandwidth-churn} We analyze two different synthetic churns. In the first one we kill one node per minute. In the second one, we still kill one node per minute, but we immediately create a new one. For \jgroups we ran the benchmarks once without killing the coordinator and once killing it.

We can see that the churn doesn't affect  \epto at all when there are only nodes leaving. We have small peaks when adding a node to \SI{3}{\mbps} or less. Probably due to running the PSS initialization method on top of having one more node spreading rumors in the system. This is confirmed at the end of the plot where \epto goes back to a normal Bandwidth after stabilization.

On the other hand, when killing the coordinator in \jgroups we can see a huge spike in bandwidth, going from \SI{1.2}{\mbps} to more than \SI{15}{\mbps}. This is due to how \jgroups operates when selecting a new coordinator.

Even when not killing the coordinator, \jgroups suffers from the churn. We can see that each time the view changes, it generates an almost 100\% increase in bandwidth usage. This is due to \jgroups having to update the view and propagate it to every peer.

\subsection{Total GigaBytes sent/received}
In all cases, \jgroups receives more bytes than it sends. This is due to the fact that we do not measure the bandwidth on the TCPGOSSIP, which is in charge of view changes.
\par
\begin{table*}[hpt]
	\centering
	\caption{Total \si{\giga\byte} sent/received in a stable system}
	\sisetup{table-format=2.2, separate-uncertainty, table-figures-uncertainty = 2, table-align-uncertainty}
	\begin{tabular}{llSSS}
		\toprule
		&& \multicolumn{3}{c}{Cluster parameters} \\
		\cmidrule{3-5}
		{Protocol}&& {$(50,50)$} & {$(50,100)$} & {$(100,50)$} \\
		\midrule
		\multirow{2}{*}{\epto}&{Receive}& 10.84(016) & 22.31(039) & 26.01(027) \\
						    &{Sending}& 10.84(016) & 22.31(039) & 26.01(027)\\
		\midrule
		\multirow{2}{*}{\jgroups}&{Receive}& 0.78(003) & 1.45(001) & 1.88(001)\\
							   &{Sending}& 0.77(003) & 1.44(001) & 1.84(001)\\
		\bottomrule
	\end{tabular}
	\label{table:total-bandwidth} 
\end{table*}

%\begin{figure}[h]
%	\centering
%	\begin{tikzpicture}
\usetikzlibrary{plotmarks}
\pgfplotsset{width=\linewidth, height=4.7cm}
\begin{groupplot}[
group style={
	group size=3 by 1,
	vertical sep=0pt,
	horizontal sep=2mm,
	xlabels at=edge bottom,
	ylabels at=edge left,
	xticklabels at=edge bottom,
	yticklabels at=edge left,
},
ymin=0,
ymax=35,
width=\linewidth / 2.5,
enlarge x limits=0.3,
ybar=0,
/pgf/bar width=3mm,
/pgfplots/area legend,
nodes near coords,
legend style={
	anchor=north west,
	at={(0.3,0.97)},
	cells={anchor=west},
	draw=none,
},
every node near coord/.append style={
	rotate=90,
	anchor=north,
	font=\tiny,
	xshift=3mm,
	yshift=0.3mm,
},
ymajorgrids,
xtick=data,
xlabel=Protocol,
ylabel={Total $\left[\si{\giga\byte}\right]$},
symbolic x coords={EpTO, JGroups},
]
\nextgroupplot
% 50-1sec
\addplot+[mark=none, pattern=north east lines,pattern color=blue, error bars/.cd,y dir=both, y explicit] coordinates {
	% Receive
	(EpTO, 10.8454953841) +- (0, 0.158556056880199)
	(JGroups, 0.8112389025) +- (0, 0.0275185471028981)
};
\addplot+[mark=none, pattern=crosshatch dots,pattern color=red, error bars/.cd,y dir=both, y explicit] coordinates {
	% Send
	(EpTO, 10.84508943053) +- (0, 0.158553532839913)
	(JGroups, 0.7979795057) +- (0, 0.0275124778152395)
};
\legend{receive, send}
\nextgroupplot
% 50-2sec
\addplot+[mark=none, pattern=north east lines,pattern color=blue, error bars/.cd,y dir=both, y explicit] coordinates {
	% Receive
	(EpTO, 22.3122950339) +- (0, 0.387040306037242)
	(JGroups, 1.481260734) +- (0, 0.0112693613185708)
};
\addplot+[mark=none, pattern=crosshatch dots,pattern color=red, error bars/.cd,y dir=both, y explicit] coordinates {
	% Send
	(EpTO, 22.3118855827) +- (0, 0.387040491969926)
	(JGroups, 1.4680794813) +- (0, 0.0112627440922006)
};
\nextgroupplot
% 100-1sec
\addplot+[mark=none, pattern=north east lines,pattern color=blue, error bars/.cd,y dir=both, y explicit] coordinates {
	% Receive
	(EpTO, 26.0194016553) +- (0, 0.2773178289900919968)
	(JGroups, 2.0044813206) +- (0, 0.0103933483299930001)
};
\addplot+[mark=none, pattern=crosshatch dots,pattern color=red, error bars/.cd,y dir=both, y explicit] coordinates {
	% Send
	(EpTO, 26.0161708501) +- (0, 0.277318407393540023)
	(JGroups, 1.9504773287) +- (0, 0.0104754176417391)
};
\end{groupplot}
%
\node[anchor=south] at (group c1r1.north) {50-1sec};
\node[anchor=south] at (group c2r1.north) {50-2sec};
\node[anchor=south] at (group c3r1.north) {100-1sec};
\end{tikzpicture}
%	\vspace{-2mm} 
%	\caption{Total bytes sent/received during an average experiment}
%	\vspace{-2mm} 
%	\label{fig:total-bandwidth}
%\end{figure}
In \autoref{table:total-bandwidth}, \epto has a worse baseline than \jgroups. This is expected as \epto sends $c*n*\log_2 n$ messages per events and \jgroups sends at least $n$ messages per event so we should have at least $c*\log_2 n$ more messages sent in \epto if \jgroups is perfect. Here we are well within this ratio.
\begin{table*}[hpt]
	\centering
	\caption{Total \si{\giga\byte} sent/received with a synthetic churn}
	\sisetup{table-format=2.2, separate-uncertainty, table-figures-uncertainty = 2, table-align-uncertainty}
	\begin{tabular}{lSSS}
		\toprule
		&& \multicolumn{2}{c}{Churn parameters} \\
		\cmidrule{3-4}
		{Protocol}&& {1 kill/minute} & {1\{kill,add\}/minute} \\
		\midrule
		\multirow{2}{*}{\epto}&{Receive}& 21.00(024) & 26.32(032)\\
							&{Sending}& 21.21(025) & 26.57(032)\\
		\midrule
		\multirow{2}{*}{\jgroups-coord}&{Receive}& 1.47(002) & 1.75(002)\\&{Sending}& 1.43(002) & 1.70(002)\\
		\midrule
		\multirow{2}{*}{\jgroups-nocoord}&{Receive}& 1.45(001) & 1.73(002)\\&{Sending}& 1.41(001) & 1.68(002)\\
		\bottomrule
	\end{tabular}
	\label{table:total-bandwidth-churn} 
\end{table*}

%\begin{figure}[h]
%	\centering
%	\begin{tikzpicture}
\usetikzlibrary{plotmarks}
\pgfplotsset{width=\linewidth, height=4.7cm}
\begin{groupplot}[
group style={
	group size=2 by 1,
	vertical sep=0pt,
	horizontal sep=2mm,
	xlabels at=edge bottom,
	ylabels at=edge left,
	xticklabels at=edge bottom,
	yticklabels at=edge left,
},
ymin=0,
ymax=35,
width=\linewidth / 1.75,
enlarge x limits=0.3,
ybar=0,
/pgf/bar width=3mm,
/pgfplots/area legend,
nodes near coords,
legend style={
	anchor=north west,
	at={(0.6,0.97)},
	cells={anchor=west},
	draw=none,
},
every node near coord/.append style={
	rotate=90,
	anchor=north,
	font=\tiny,
	xshift=3mm,
	yshift=0.3mm,
},
ymajorgrids,
xtick=data,
xlabel=Protocol,
xticklabel style={text height=1.5ex, rotate=30}, 
ylabel={Total $\left[\si{\giga\byte}\right]$},
symbolic x coords={EpTO, JGroups-nocoord, JGroups-coord},
]
\nextgroupplot
% 100-1sec 1kill/min
\addplot+[mark=none, pattern=north east lines,pattern color=blue, error bars/.cd,y dir=both, y explicit] coordinates {
	% Receive
	(EpTO, 21.001445292) +- (0, 0.248116313343146)
	(JGroups-nocoord, 1.5381357651) +- (0, 0.0115143286137117)
	(JGroups-coord, 1.5510862161) +- (0, 0.0181726968637206)
};
\addplot+[mark=none, pattern=crosshatch dots,pattern color=red, error bars/.cd,y dir=both, y explicit] coordinates {
	% Send
	(EpTO, 21.216776898) +- (0, 0.255744716454951)
	(JGroups-nocoord, 1.4935079775) +- (0, 0.0119366849923253)
	(JGroups-coord, 1.5064716889) +- (0, 0.0186384151121228)
};
\legend{receive, send}
\nextgroupplot
% 100-1sec 1kill, 1add/min
\addplot+[mark=none, pattern=north east lines,pattern color=blue, error bars/.cd,y dir=both, y explicit] coordinates {
	% Receive
	(EpTO, 26.325256447) +- (0, 0.320079175714576)
	(JGroups-nocoord, 1.8338254834) +- (0, 0.0187827622384514)
	(JGroups-coord, 1.8511940589) +- (0, 0.0183913437417294)
};
\addplot+[mark=none, pattern=crosshatch dots,pattern color=red, error bars/.cd,y dir=both, y explicit] coordinates {
	% Send
	(EpTO, 26.5733934455) +- (0, 0.320312180261336)
	(JGroups-nocoord, 1.7777051021) +- (0, 0.0190481557873808)
	(JGroups-coord, 1.7959073697) +- (0, 0.0189017779847593)
};
\end{groupplot}
%
\node[anchor=south] at (group c1r1.north) {1 kill/minute};
\node[anchor=south] at (group c2r1.north) {1 kill/minute, 1 add/minute};
\end{tikzpicture}
%	\vspace{-2mm} 
%	\caption{Total bytes sent/received during an average experiment with churn}
%	\vspace{-2mm} 
%	\label{fig:total-bandwidth-churn}
%\end{figure}
In \autoref{table:total-bandwidth-churn} we see that \jgroups total bandwidth usage is smaller when there is churn. One hypothesis for this is that a JGroups replica takes a longer time to start up compared to stopping a replica. Therefore the overall benchmark has a longer time with less than 100 replicas. We also do not see a difference whether we kill the coordinator or not. This can be explained by the fact that before the detection of the faulty coordinator \jgroups is forced to a stop for time up to \SI{20}{\second}. The big spike afterwards compensates for this hole.
\subsection{Local Times}
\label{sub:local-times}
\begin{figure*}[hpt]
	\centering
	\begin{tikzpicture}
\begin{groupplot}[
group style={
	group size=3 by 1,
	vertical sep=0pt,
	horizontal sep=6mm,
	xlabels at=edge bottom,
	ylabels at=edge left,
	xticklabels at=edge bottom,
	yticklabels at=edge left,
},
every axis plot/.append style={very thick},
width=\linewidth / 3,
height=5cm,
grid=major,
grid style={dashed,gray!30},
ymax=1.05,
ymin=0,
xmin=1185,
xmax=1202.5,
xlabel={Time $\left[\si{\second}\right]$},
x tick label style={/pgf/number format/fixed},
ytick={0,0.2,0.4,0.6,0.8,1},
yticklabels={0,0.2,0.4,0.6,0.8,1},
ylabel=Percentiles,
legend columns=3,
legend cell align=left,
legend style={at={(1.6,1.4)},anchor=north, font=\small, draw=none},
cycle list name=my colors,
]
\nextgroupplot
\addplot+[const plot, mark=none] table[x=EpTO-50-1sec-x, y=EpTO-50-1sec-y, col sep=comma] {\tablelocaltime};
\addplot+[const plot, mark=none] table[x=JGroups-50-1sec-x, y=JGroups-50-1sec-y, col sep=comma] {\tablelocaltime};
\legend{EpTO, JGroups}
%
\nextgroupplot
\addplot+[const plot, mark=none] table[x=EpTO-50-2sec-x, y=EpTO-50-2sec-y, col sep=comma] {\tablelocaltime};
\addplot+[const plot, mark=none] table[x=JGroups-50-2sec-x, y=JGroups-50-2sec-y, col sep=comma] {\tablelocaltime};
%
\nextgroupplot
\addplot+[const plot, mark=none] table[x=EpTO-100-1sec-x, y=EpTO-100-1sec-y, col sep=comma] {\tablelocaltime};
\addplot+[const plot, mark=none] table[x=JGroups-100-1sec-x, y=JGroups-100-1sec-y, col sep=comma] {\tablelocaltime};
\end{groupplot}
%
\node[anchor=south] at (group c1r1.north) {$(50,50)$};
\node[anchor=south] at (group c2r1.north) {$(50,100)$};
\node[anchor=south] at (group c3r1.north) {$(100,50)$};
\end{tikzpicture}
	\vspace{-2mm} 
	\caption{Local dissemination times}
	\vspace{-2mm}
	\label{fig:local-times} 
\end{figure*}

\begin{figure}[hpt]
	\centering
	\begin{tikzpicture}
\begin{groupplot}[
group style={
group size=2 by 1,
vertical sep=0pt,
horizontal sep=6mm,
xlabels at=edge bottom,
ylabels at=edge left,
xticklabels at=edge bottom,
yticklabels at=edge left,
},
every axis plot/.append style={very thick},
height=5cm, width=\linewidth / 1.75,
grid=major,
grid style={dashed,gray!30},
ymax=1.05,
ymin=0,
xmin=1185,
xmax=1202.5,
xlabel={Time $\left[\si{\second}\right]$},
x tick label style={/pgf/number format/fixed},
ytick={0,0.2,0.4,0.6,0.8,1},
yticklabels={0,0.2,0.4,0.6,0.8,1},
ylabel=Percentiles,
legend columns=3,
legend cell align=left,
legend style={at={(1.1,1.3)},anchor=north, font=\small, draw=none},
cycle list name=my colors,
]
\nextgroupplot
\addplot+[const plot, mark=none] table[x=EpTO-suspend-x, y=EpTO-suspend-y, col sep=comma] {\tablelocaltime};
\addplot+[const plot, mark=none] table[x=JGroups-suspend-coord-x, y=JGroups-suspend-coord-y, col sep=comma] {\tablelocaltime};
\addplot+[const plot, mark=nonee] table[x=JGroups-suspend-nocoord-x, y=JGroups-suspend-nocoord-y, col sep=comma] {\tablelocaltime};
\legend{\epto, \jgroups-nocoord, \jgroups-coord}
%
\nextgroupplot
\addplot+[const plot, mark=nonek] table[x=EpTO-add-suspend-x, y=EpTO-add-suspend-y, col sep=comma] {\tablelocaltime};
\addplot+[const plot, mark=none] table[x=JGroups-add-suspend-coord-x, y=JGroups-add-suspend-coord-y, col sep=comma] {\tablelocaltime};
\addplot+[const plot, mark=none] table[x=JGroups-add-suspend-nocoord-x, y=JGroups-add-suspend-nocoord-y, col sep=comma] {\tablelocaltime};
\end{groupplot}
%
\node[anchor=south] at (group c1r1.north) {1 kill/minute};
\node[anchor=south] at (group c2r1.north) {1 \{kill,add\}/minute};
\end{tikzpicture}
	\vspace{-2mm} 
	\caption{Local dissemination times with churn}
	\vspace{-2mm} 
	\label{fig:local-times-churn} 
\end{figure}
In \autoref{fig:local-times}, \jgroups delivers all events quicker than \epto in all scenarios, even when churn is involved as is shown in \autoref{fig:local-times-churn}. However, \epto is not too far behind. The difference between \epto and \jgroups is likely to be even smaller when running them in a real WAN network due to the latency. \epto in our configuration has a $\delta$ period of \SI{100}{\milli\second} and is thus handicapped against \jgroups in a LAN environment, because it only increments the TTL of an event every \SI{100}{\milli\second}. We expect \epto to outperform \jgroups in this situation when we have a high number of peers in the system.
\subsection{Global Times}
\begin{figure*}[hpt]
	\centering
	\begin{tikzpicture}
\begin{groupplot}[
group style={
	group size=3 by 1,
	vertical sep=0pt,
	horizontal sep=6mm,
	xlabels at=edge bottom,
	ylabels at=edge left,
	xticklabels at=edge bottom,
	yticklabels at=edge left,
},
every axis plot/.append style={very thick},
width=\linewidth / 3,
height=5cm,
grid=major,
grid style={dashed,gray!30},
ymax=1.05,
ymin=0,
xmin=1185,
xmax=1202.5,
xlabel={Time $\left[\si{\second}\right]$},
x tick label style={/pgf/number format/fixed},
ytick={0,0.2,0.4,0.6,0.8,1},
yticklabels={0,0.2,0.4,0.6,0.8,1},
ylabel=Percentiles,
legend columns=3,
legend cell align=left,
legend style={at={(1.6,1.3)},anchor=north, font=\small, draw=none, fill=none},
cycle list name=my colors,
]
\nextgroupplot
\addplot+[const plot, mark=none] table[x=EpTO-50-1sec-x, y=EpTO-50-1sec-y, col sep=comma] {\tableglobaltime};
\addplot+[const plot, mark=none] table[x=JGroups-50-1sec-x, y=JGroups-50-1sec-y, col sep=comma] {\tableglobaltime};
\legend{\epto, \jgroups}
%
\nextgroupplot
\addplot+[const plot, mark=none] table[x=EpTO-50-2sec-x, y=EpTO-50-2sec-y, col sep=comma] {\tableglobaltime};
\addplot+[const plot, mark=none] table[x=JGroups-50-2sec-x, y=JGroups-50-2sec-y, col sep=comma] {\tableglobaltime};
%
\nextgroupplot
\addplot+[const plot, mark=none] table[x=EpTO-100-1sec-x, y=EpTO-100-1sec-y, col sep=comma] {\tableglobaltime};
\addplot+[const plot, mark=none] table[x=JGroups-100-1sec-x, y=JGroups-100-1sec-y, col sep=comma] {\tableglobaltime};
\end{groupplot}
%
\node[anchor=south] at (group c1r1.north) {$(50,50)$};
\node[anchor=south] at (group c2r1.north) {$(50,100)$};
\node[anchor=south] at (group c3r1.north) {$(100,50)$};
\end{tikzpicture}
	\vspace{-2mm} 
	\caption{Global dissemination times}
	\vspace{-2mm}
	\label{fig:global-times}  
\end{figure*}

\begin{figure}[hpt]
	\centering
	\begin{tikzpicture}
\begin{groupplot}[
group style={
group size=2 by 1,
vertical sep=0pt,
horizontal sep=6mm,
xlabels at=edge bottom,
ylabels at=edge left,
xticklabels at=edge bottom,
yticklabels at=edge left,
},
every axis plot/.append style={very thick},
height=5cm, width=\linewidth / 1.75,
grid=major,
grid style={dashed,gray!30},
ymax=1.05,
ymin=0,
xmin=1185,
xmax=1202.5,
xlabel={Time $\left[\si{\second}\right]$},
x tick label style={/pgf/number format/fixed},
ytick={0,0.2,0.4,0.6,0.8,1},
yticklabels={0,0.2,0.4,0.6,0.8,1},
ylabel=Percentiles,
legend columns=3,
legend cell align=left,
legend style={at={(1.1,1.3)},anchor=north, font=\small, draw=none},
cycle list name=my colors,
]
\nextgroupplot
\addplot+[const plot, mark=none] table[x=EpTO-suspend-x, y=EpTO-suspend-y, col sep=comma] {\tableglobaltime};
\addplot+[const plot, mark=none] table[x=JGroups-suspend-coord-x, y=JGroups-suspend-coord-y, col sep=comma] {\tableglobaltime};
\addplot+[const plot, mark=nonee] table[x=JGroups-suspend-nocoord-x, y=JGroups-suspend-nocoord-y, col sep=comma] {\tableglobaltime};
\legend{EpTO, JGroups-nocoord, JGroups-coord}
%
\nextgroupplot
\addplot+[const plot, mark=nonek] table[x=EpTO-add-suspend-x, y=EpTO-add-suspend-y, col sep=comma] {\tableglobaltime};
\addplot+[const plot, mark=none] table[x=JGroups-add-suspend-coord-x, y=JGroups-add-suspend-coord-y, col sep=comma] {\tableglobaltime};
\addplot+[const plot, mark=none] table[x=JGroups-add-suspend-nocoord-x, y=JGroups-add-suspend-nocoord-y, col sep=comma] {\tableglobaltime};
\end{groupplot}
%
\node[anchor=south] at (group c1r1.north) {1 kill/minute};
\node[anchor=south] at (group c2r1.north) {1 \{kill,add\}/minute};
\end{tikzpicture}
	\vspace{-2mm} 
	\caption{Global dissemination times with churn}
	\vspace{-2mm} 
	\label{fig:global-times-churn} 
\end{figure}
We computed global times as well. They are represented in \autoref{fig:global-times} and \autoref{fig:global-times-churn}. These global times are of less interest than their local counterpart as the differences in clocks between hosts can skew this measurement.

Nonetheless, here too we can see that \epto is consistently slower than \jgroups for the same reason as stated in \autoref{sub:local-times}.
\subsection{Local Dissemination stretch}
\begin{figure*}[hpt]
	\centering
	\begin{tikzpicture}
\begin{groupplot}[
group style={
	group size=3 by 1,
	vertical sep=0pt,
	horizontal sep=4mm,
	xlabels at=edge bottom,
	ylabels at=edge left,
	xticklabels at=edge bottom,
	yticklabels at=edge left,
},
every axis plot/.append style={very thick},
width=\linewidth / 2.5,
height=5cm,
grid=major,
grid style={dashed,gray!30},
ymax=1.05,
ymin=0,
xmax=3.5,
xlabel={Time $\left[\si{\second}\right]$},
x tick label style={/pgf/number format/fixed},
ytick={0,0.2,0.4,0.6,0.8,1},
yticklabels={0,0.2,0.4,0.6,0.8,1},
ylabel=Percentiles,
legend columns=3,
legend cell align=left,
legend style={at={(1.6,1.3)},anchor=north, font=\small, draw=none},
cycle list name=my colors,
]
\nextgroupplot
\addplot+[const plot, mark=none] table[x=EpTO-50-1sec-x, y=EpTO-50-1sec-y, col sep=comma] {\tablelocaldeltas};
\addplot+[const plot, mark=none] table[x=JGroups-50-1sec-x, y=JGroups-50-1sec-y, col sep=comma] {\tablelocaldeltas};
\legend{\epto, \jgroups}
%
\nextgroupplot
\addplot+[const plot, mark=none] table[x=EpTO-50-2sec-x, y=EpTO-50-2sec-y, col sep=comma] {\tablelocaldeltas};
\addplot+[const plot, mark=none] table[x=JGroups-50-2sec-x, y=JGroups-50-2sec-y, col sep=comma] {\tablelocaldeltas};
%
\nextgroupplot
\addplot+[const plot, mark=none] table[x=EpTO-100-1sec-x, y=EpTO-100-1sec-y, col sep=comma] {\tablelocaldeltas};
\addplot+[const plot, mark=none] table[x=JGroups-100-1sec-x, y=JGroups-100-1sec-y, col sep=comma] {\tablelocaldeltas};
\end{groupplot}
%
\node[anchor=south] at (group c1r1.north) {$(50,50)$};
\node[anchor=south] at (group c2r1.north) {$(50,100)$};
\node[anchor=south] at (group c3r1.north) {$(100,50)$};
\end{tikzpicture}
	\vspace{-2mm} 
	\caption{Local dissemination stretch}
	\vspace{-2mm}
	\label{fig:local-delta}  
\end{figure*}
In \autoref{fig:local-delta}, We see the percentiles of the local dissemination stretch. The local dissemination stretch is the time measurement between the sending of an event by a peer and the delivery of this event locally.
\par
\jgroups is usually much faster than \epto in a perfect environment. This is expected as the benchmarks involve a small number of nodes and are performed in a LAN environment with minimal latency. The median dissemination stretch of \jgroups is around \SI{7}{\milli\second} where as the median dissemination stretch of \epto is around \SI{630}{\milli\second} for $(100,50)$. When increasing the number of peers, \jgroups starts to have longer delivery times for some outliers. A small portion of the local dissemination stretches are really fast (\SI{1}{\milli\second} or lower). This happens when the coordinator itself sends an event. Since it has to deliver it to himself the local dissemination stretch is intra-process and thus extremely fast.

\begin{figure}[hpt]
	\centering
	\begin{tikzpicture}
\begin{groupplot}[
group style={
	group size=2 by 1,
	vertical sep=0pt,
	horizontal sep=6mm,
	xlabels at=edge bottom,
	ylabels at=edge left,
	xticklabels at=edge bottom,
	yticklabels at=edge left,
},
every axis plot/.append style={very thick},
height=5cm, width=\linewidth / 1.75,
grid=major,
grid style={dashed,gray!30},
ymax=1.05,
ymin=0,
xmax=25,
scaled x ticks = false,
x tick label style={/pgf/number format/fixed},
xlabel={Time $\left[\si{\second}\right]$},
ytick={0,0.2,0.4,0.6,0.8,1},
yticklabels={0,0.2,0.4,0.6,0.8,1},
ylabel=Percentiles,
legend columns=3,
legend cell align=left,
legend style={at={(1.1,1.3)},anchor=north, font=\small, draw=none},
cycle list name=my colors,
]
\nextgroupplot
\addplot+[const plot, mark=none] table[x=EpTO-suspend-x, y=EpTO-suspend-y, col sep=comma] {\tablelocaldeltas};
\addplot+[const plot, mark=none] table[x=JGroups-suspend-coord-x, y=JGroups-suspend-coord-y, col sep=comma] {\tablelocaldeltas};
\addplot+[const plot, mark=nonee] table[x=JGroups-suspend-nocoord-x, y=JGroups-suspend-nocoord-y, col sep=comma] {\tablelocaldeltas};
\legend{\epto, \jgroups-coord, \jgroups-nocoord}
%
\nextgroupplot
\addplot+[const plot, mark=none] table[x=EpTO-add-suspend-x, y=EpTO-add-suspend-y, col sep=comma] {\tablelocaldeltas};
\addplot+[const plot, mark=none] table[x=JGroups-add-suspend-coord-x, y=JGroups-add-suspend-coord-y, col sep=comma] {\tablelocaldeltas};
\addplot+[const plot, mark=none] table[x=JGroups-add-suspend-nocoord-x, y=JGroups-add-suspend-nocoord-y, col sep=comma] {\tablelocaldeltas};
\end{groupplot}
%
\node[anchor=south] at (group c1r1.north) {1 kill/minute};
\node[anchor=south] at (group c2r1.north) {1 \{kill,add\}/minute};
\end{tikzpicture}
	\vspace{-2mm} 
	\caption{Local dissemination stretch with churn}
	\vspace{-2mm}
	\label{fig:local-delta-churn}   
\end{figure}
In \autoref{fig:local-delta-churn} We see a completely different picture. When under churn, the 95th percentile of \jgroups is at \SI{31}{\milli\second} compared to \SI{14}{\milli\second} when there is no churn. The highest percentiles are at more than \SI{10}{\second}. This effect is due to the coordinator dying as we clearly see that it does not happen when we do not kill it.

The median is bigger at around \SI{9}{\milli\second}, whether we kill the coordinator or not. This shows that there are some degradation in \jgroups local dissemination stretch when under churn.

On the contrary, \epto performs very well under churn. The median degrade a bit at \SI{650}{\milli\second} with the 99th percentile being at \SI{1030}{\milli\second} compared to \SI{982}{\milli\second} when no churn is happening.
\subsection{Events sent}
The variation observed in the different tests are due to the fact that we run an experiment for a period of time and that in some configurations a randomness decides whether an event is sent or not.
\begin{table}[hpt]
	\centering
	\caption{Total events sent in a stable environment}
\sisetup{table-format=6.1, separate-uncertainty, table-figures-uncertainty = 2, table-align-uncertainty}
\begin{tabular}{lSSS}
	\toprule
	& \multicolumn{3}{c}{Cluster parameters} \\
	\cmidrule{2-4}
	Protocol & {$(50,50)$} & {$(50,100)$} & {$(100,50)$} \\
	\midrule
	\epto & 59993.8(33) & 119898.2(97) & 59913.0(1643) \\
	\jgroups & 59961.9(109) & 119885.7(50) & 60023.1(2871) \\
	\bottomrule
\end{tabular}
\label{table:total-events}  
\end{table}

%\begin{figure}[h]
%	\centering
%	\begin{tikzpicture}
\usetikzlibrary{plotmarks}
\pgfplotsset{width=\linewidth, height=4.7cm}
\begin{groupplot}[
group style={
	group size=3 by 1,
	vertical sep=0pt,
	horizontal sep=2mm,
	xlabels at=edge bottom,
	ylabels at=edge left,
	xticklabels at=edge bottom,
	yticklabels at=edge left,
},
ymin=40000,
ymax=155000,
width=\linewidth / 2.5,
enlarge x limits=0.3,
ybar=0,
/pgf/bar width=3mm,
/pgfplots/area legend,
nodes near coords,
legend style={
	anchor=north west,
	at={(0.3,0.97)},
	cells={anchor=west},
	draw=none,
	fill=none,
},
every node near coord/.append style={
	rotate=90,
	anchor=north,
	font=\tiny,
	xshift=5mm,
	yshift=1mm,
},
ymajorgrids,
xtick=data,
xlabel=Protocol,
ylabel={Events sent},
scaled ticks=false, tick label style={/pgf/number format/fixed},
symbolic x coords={EpTO, JGroups},
]
\nextgroupplot
% 50-1sec
\addplot[mark=none, pattern=north east lines, error bars/.cd,y dir=both, y explicit] coordinates {
	% Receive
	(EpTO, 59993.8) +- (0, 3.32665998663345)
	(JGroups, 59961.9) +- (0, 10.9285558667798)
};
%
\nextgroupplot
% 50-2sec
\addplot[mark=none, pattern=north east lines, error bars/.cd,y dir=both, y explicit] coordinates {
	% Receive
	(EpTO, 119898.2) +- (0, 9.75021367287237)
	(JGroups, 119885.7) +- (0, 5.01220732035492)
};
\nextgroupplot
% 100-1sec
\addplot[mark=none, pattern=north east lines, error bars/.cd,y dir=both, y explicit] coordinates {
	% Receive
	(EpTO, 59913) +- (0, 164.332319131421)
	(JGroups, 60023.1) +- (0, 287.150928181603)
};
\end{groupplot}
%
\node[anchor=south] at (group c1r1.north) {50-1sec};
\node[anchor=south] at (group c2r1.north) {50-2sec};
\node[anchor=south] at (group c3r1.north) {100-1sec};
\end{tikzpicture}
%	\vspace{-2mm} 
%	\caption{Total events sent per experiment on average}
%	\vspace{-2mm}
%	\label{fig:total-events}   
%\end{figure}
In \autoref{table:total-events} we see that both \epto and \jgroups deliver the same amount of events. This is expected in a perfect environment.
\begin{table}[hpt]
	\centering
	\caption{Total events sent with a synthetic churn}
\sisetup{table-format=6.1, separate-uncertainty, table-figures-uncertainty = 2, table-align-uncertainty}
\begin{tabular}{lSSS}
	\toprule
	& \multicolumn{2}{c}{Cluster parameters} \\
	\cmidrule{2-3}
	Protocol & {1 kill/minute} & {1\{kill,add\}/minute} \\
	\midrule
	\epto & 53898.5(1339) & 59798.6(1401) \\
	\jgroups-coord & 53834.7(1755) & 59507.9(2409) \\
	\jgroups-nocoord & 53830.5(2003) & 59450.5(1751) \\
	\bottomrule
\end{tabular}
    \label{table:total-events-churn}
\end{table}
%\begin{figure}[h]
%	\centering
%	\begin{tikzpicture}
\pgfplotsset{width=\linewidth, height=4.7cm}
\begin{groupplot}[
group style={
	group size=2 by 1,
	vertical sep=0pt,
	horizontal sep=2mm,
	xlabels at=edge bottom,
	ylabels at=edge left,
	xticklabels at=edge bottom,
	yticklabels at=edge left,
},
ymin=53000,
ymax=65000,
width=\linewidth / 1.75,
enlarge x limits=0.3,
ybar=0,
/pgf/bar width=3mm,
/pgfplots/area legend,
nodes near coords,
legend style={
	anchor=north west,
	at={(0.6,0.97)},
	cells={anchor=west},
	draw=none,
	fill=none,
},
every node near coord/.append style={
	rotate=90,
	anchor=north,
	font=\tiny,
	xshift=5mm,
	yshift=1mm,
},
ymajorgrids,
xtick=data,
xlabel=Protocol,
xticklabel style={text height=1.5ex, rotate=30}, 
scaled ticks=false, tick label style={/pgf/number format/fixed},
ylabel={Events sent},
symbolic x coords={EpTO, JGroups-nocoord, JGroups-coord},
]
\nextgroupplot
% 100-1sec 1kill/min
\addplot[mark=none, pattern=north east lines, error bars/.cd,y dir=both, y explicit] coordinates {
	% Receive
	(EpTO, 53898.5) +- (0, 133.952935350029)
	(JGroups-nocoord, 53830.5) +- (0, 200.355933278752)
	(JGroups-coord, 53834.7) +- (0, 175.463861426411)
};
%
\nextgroupplot
% 100-1sec 1kill, 1add/min
\addplot[mark=none, pattern=north east lines, error bars/.cd,y dir=both, y explicit] coordinates {
	% Receive
	(EpTO, 59798.6) +- (0, 140.107260498678)
	(JGroups-nocoord, 59450.5) +- (0, 175.171820412607)
	(JGroups-coord, 59507.9) +- (0, 240.914669079693)
};
\end{groupplot}
%
\node[anchor=south] at (group c1r1.north) {1 kill/minute};
\node[anchor=south] at (group c2r1.north) {1 \{kill,add\}/minute};
\end{tikzpicture}
%	\vspace{-2mm} 
%	\caption{Total events sent per experiment during churn on average}
%	\vspace{-2mm}
%	\label{fig:total-events-churn}  
%\end{figure}

In \autoref{table:total-events-churn} When only killing nodes, \epto and \jgroups again deliver the same amount of events. When killing and adding nodes, JGroups seems to deliver a smaller amount of nodes, however it does not look significant enough to draw any conclusion. \jt{I didn't run any statistical analysis}
\subsection{Real Trace}
\begin{figure}[hpt]
	\centering
	\begin{tikzpicture}
\usetikzlibrary{plotmarks}
\pgfplotsset{
	height=4cm,
	width=\linewidth,
	every axis plot post/.append style={
		solid,
		very thin,
		mark=none
	},
	/pgfplots/area cycle list/.style={/pgfplots/cycle list={%
			{black,fill=black,mark=none},%
			{black,fill=white!25!black,mark=none},%
			{black,fill=white!50!black,mark=none},%
			{black,fill=white!75!black,mark=none},%
			{black,fill=white,mark=none},%
		}
	},
}
\begin{groupplot}[
ymajorgrids,
group style={
	group size=1 by 4,
	vertical sep=6mm,
	horizontal sep=0mm,
	xlabels at=edge bottom,
	xticklabels at=edge bottom,
	ylabels at=edge left,
},
stack plots=y,area style, enlarge x limits=false, 
ymin=0,
xmin=0,
xmax=61,
ylabel={Bandwidth $\left[\SI{}{\mbps}\right]$},
xlabel={Time $\left[\si{\minute}\right]$},
legend columns=5,
legend cell align=left,
legend style={at={(0.2,2)},anchor=north, font=\small, draw=none},]
\nextgroupplot[height=2.2cm, 
ylabel={Nodes}, 
ymin=80,ymax=120,
ytick={80,120}, 
yticklabels={80,...,120},
xmajorgrids,
yminorgrids,
tick label style={font=\footnotesize},
label style={font=\tiny}]
\addplot[const plot, color=blue, mark=none] table [x=minute, y=size, col sep=comma] {csv-data/real-trace.csv};
\nextgroupplot[ymax=4,ytick={0,1,2,3,4},
yticklabels={0,1,2,3,4}]
\addplot table[x=time,y=EpTO-real-trace-0.000000, col sep=comma]{\tableaverage} \closedcycle;
\addplot table[x=time,y=EpTO-real-trace-0.250000, col sep=comma]{\tableaverage} \closedcycle;
\addplot table[x=time,y=EpTO-real-trace-0.500000, col sep=comma]{\tableaverage} \closedcycle;
\addplot table[x=time,y=EpTO-real-trace-0.750000, col sep=comma]{\tableaverage} \closedcycle;
\addplot table[x=time,y=EpTO-real-trace-1.000000, col sep=comma]{\tableaverage} \closedcycle;
\legend{0, 0.25, 0.5, 0.75, 1}
%
\nextgroupplot%[ymax=4,ytick={0,1,2,3,4},]
\addplot table[x=time,y=JGroups-coord-0.000000, col sep=comma]{\tableaverage} \closedcycle;
\addplot table[x=time,y=JGroups-coord-0.250000, col sep=comma]{\tableaverage} \closedcycle;
\addplot table[x=time,y=JGroups-coord-0.500000, col sep=comma]{\tableaverage} \closedcycle;
\addplot table[x=time,y=JGroups-coord-0.750000, col sep=comma]{\tableaverage} \closedcycle;
\addplot table[x=time,y=JGroups-coord-1.000000, col sep=comma]{\tableaverage} \closedcycle;
%
\nextgroupplot[ymax=4,ytick={0,1,2,3,4}]
\addplot table[x=time, y=JGroups-nocoord-0.000000, col sep=comma]{\tableaverage} \closedcycle;
\addplot table[x=time,y=JGroups-nocoord-0.250000, col sep=comma]{\tableaverage} \closedcycle;
\addplot table[x=time,y=JGroups-nocoord-0.500000, col sep=comma]{\tableaverage} \closedcycle;
\addplot table[x=time,y=JGroups-nocoord-0.750000, col sep=comma]{\tableaverage} \closedcycle;
\addplot table[x=time,y=JGroups-nocoord-1.000000, col sep=comma]{\tableaverage} \closedcycle;
\end{groupplot}
%
\node[anchor=south, rotate=-90] at (group c1r2.east) {EpTO};
\node[anchor=south, rotate=-90] at (group c1r3.east) {JGroups coord};
\node[anchor=south, rotate=-90] at (group c1r4.east) {JGroups nocoord};
\end{tikzpicture}
	\vspace{-2mm} 
	\caption{Throughput percentiles of a node during an experiment with churn}
	\vspace{-2mm} 
	\label{fig:bandwidth-real-churn}
\end{figure}
\autoref{fig:bandwidth-real-churn} confirms the results from the synthetic churn. \epto is not affected by the churn. As before, \jgroups is affected in the same situations as when following a synthetic churn. The spikes visible on \jgroups-coord and \jgroups-nocoord show the different view changes occuring. 
\par
\begin{table}[hpt]
	\centering
	\caption{Total \si{\giga\byte} sent/received}
	\sisetup{table-format=2.2, separate-uncertainty, table-figures-uncertainty = 2, table-align-uncertainty}
	\begin{tabular}{lSS}
		\toprule
		&& \multicolumn{1}{c}{Churn parameters} \\
		\cmidrule{3-3}
		{Protocol}&& {Real Trace} \\
		\midrule
		\multirow{2}{*}{\epto}&{Receive}& 81.41(108)\\
		&{Sending}& 82.67(108)\\
		\midrule
		\multirow{2}{*}{\jgroups-coord}&{Receive}& 5.56(008)\\
		&{Sending}& 5.40(008)\\
		\midrule
		\multirow{2}{*}{\jgroups-nocoord}&{Receive}& 5.58(005)\\
		&{Sending}& 5.43(005)\\
		\bottomrule
	\end{tabular}
	\label{table:total-bandwidth-real-churn} 
\end{table}
\autoref{table:total-bandwidth-real-churn} also confirms our earlier results. We run the experiment for approximately 3 times the synthetic period and we get approximately 3 times more \SI{}{\giga\byte} used.
\par
%\begin{figure}[h]
%	\centering
%	\begin{tikzpicture}
\usetikzlibrary{plotmarks}
\pgfplotsset{width=\linewidth, height=4.7cm}
\begin{axis}[
ymin=0,
ymax=85,
enlarge x limits=0.3,
ybar=0,
/pgf/bar width=3mm,
/pgfplots/area legend,
nodes near coords,
legend style={
	anchor=north west,
	at={(0.6,0.97)},
	cells={anchor=west},
	draw=none,
	fill=none,
},
every node near coord/.append style={
	rotate=90,
	anchor=north,
	font=\tiny,
	xshift=3mm,
	yshift=0.3mm,
},
ymajorgrids,
xtick=data,
xlabel=Protocol,
xticklabel style={text height=1.5ex, rotate=30}, 
ylabel={Total $\left[\si{\giga\byte}\right]$},
symbolic x coords={EpTO, JGroups-nocoord, JGroups-coord},
]
% 100-1sec 1kill/min
\addplot+[mark=none, pattern=north east lines,pattern color=blue, error bars/.cd,y dir=both, y explicit] coordinates {
	% Receive
	(EpTO, 81.4132665062) +- (0, 1.0822819001435)
	(JGroups-nocoord, 5.6382174422) +- (0, 0.058029870294168)
	(JGroups-coord, 5.6118450728) +- (0, 0.089291024953546)
};
\addplot+[mark=none, pattern=crosshatch dots,pattern color=red, error bars/.cd,y dir=both, y explicit] coordinates {
	% Send
	(EpTO, 82.6741897528) +- (0, 1.08368411575119)
	(JGroups-nocoord, 5.4807749566) +- (0, 0.056977818312573)
	(JGroups-coord, 5.4548362802) +- (0, 0.086273738554255)
};
\legend{receive, send}
\end{axis}
\end{tikzpicture}
%	\vspace{-2mm} 
%	\caption{Total bytes sent/received during an average experiment with churn}
%	\vspace{-2mm} 
%	\label{fig:total-bandwidth-real-churn}
%\end{figure}
\begin{figure}[hpt]
	\centering
	\begin{tikzpicture}
\begin{axis}[
every axis plot/.append style={very thick},
height=5cm,
grid=major,
grid style={dashed,gray!30},
ymax=1.05,
ymin=0,
xlabel={Time $\left[\si{\second}\right]$},
x tick label style={/pgf/number format/fixed},
ytick={0,0.2,0.4,0.6,0.8,1},
yticklabels={0,0.2,0.4,0.6,0.8,1},
xmin=3649,
xmax=3663,
xtick={3650,3655,3660},
xticklabels={3650,3655,3660},
ylabel=Percentiles,
legend columns=3,
legend cell align=left,
legend style={at={(0.5,1.2)},anchor=north, font=\small, draw=none},
cycle list name=my colors,
]
\addplot+[const plot, mark=none] table[x=EpTO-real-trace-x, y=EpTO-real-trace-y, col sep=comma] {\tablelocaltime};
\addplot+[const plot, mark=none] table[x=JGroups-coord-x, y=JGroups-coord-y, col sep=comma] {\tablelocaltime};
\addplot+[const plot, mark=nonee] table[x=JGroups-nocoord-x, y=JGroups-nocoord-y, col sep=comma] {\tablelocaltime};
\legend{\epto, \jgroups-nocoord, \jgroups-coord}
\end{axis}
\end{tikzpicture}
	\vspace{-2mm} 
	\caption{Local dissemination times}
	\vspace{-2mm}
	\label{fig:local-times-real-churn} 
\end{figure}
\begin{figure}[hpt]
	\centering
	\begin{tikzpicture}
\begin{axis}[
every axis plot/.append style={very thick},
height=5cm,
grid=major,
grid style={dashed,gray!30},
ymax=1.05,
ymin=0,
xlabel={Time $\left[\si{\second}\right]$},
x tick label style={/pgf/number format/fixed},
ytick={0,0.2,0.4,0.6,0.8,1},
yticklabels={0,0.2,0.4,0.6,0.8,1},
xmin=3649,
xmax=3663,
xtick={3650,3655,3660},
xticklabels={3650,3655,3660},
ylabel=Percentiles,
legend columns=3,
legend cell align=left,
legend style={at={(0.5,1.2)},anchor=north, font=\small, draw=none},
cycle list name=my colors,
]
\addplot+[const plot, mark=none] table[x=EpTO-real-trace-x, y=EpTO-real-trace-y, col sep=comma] {\tableglobaltime};
\addplot+[const plot, mark=none] table[x=JGroups-coord-x, y=JGroups-coord-y, col sep=comma] {\tableglobaltime};
\addplot+[const plot, mark=nonee] table[x=JGroups-nocoord-x, y=JGroups-nocoord-y, col sep=comma] {\tableglobaltime};
\legend{\epto, \jgroups-nocoord, \jgroups-coord}
\end{axis}
\end{tikzpicture}
	\vspace{-2mm} 
	\caption{Global dissemination times with churn}
	\vspace{-2mm} 
	\label{fig:global-times-real-churn} 
\end{figure}
The local and global dissemination times shown in respectively \autoref{fig:local-times-real-churn} and \autoref{fig:global-times-real-churn} still show \jgroups to outperform \epto in this scenario. Again we want to emphasize that \epto would more than probably outperform \jgroups in a bigger size cluster.
\par
\begin{figure}[hpt]
	\centering
	\begin{tikzpicture}
\begin{axis}[
every axis plot/.append style={very thick},
height=5cm,
grid=major,
grid style={dashed,gray!30},
ymax=1.05,
ymin=0,
scaled x ticks = false,
x tick label style={/pgf/number format/fixed},
xlabel={Time $\left[\si{\second}\right]$},
ytick={0,0.2,0.4,0.6,0.8,1},
yticklabels={0,0.2,0.4,0.6,0.8,1},
ylabel=Percentiles,
legend columns=3,
legend cell align=left,
legend style={at={(0.5,1.2)},anchor=north, font=\small, draw=none},
cycle list name=my colors,
]
\addplot+[const plot, mark=none] table[x=EpTO-real-trace-x, y=EpTO-real-trace-y, col sep=comma] {\tablelocaldeltas};
\addplot+[const plot, mark=none] table[x=JGroups-coord-x, y=JGroups-coord-y, col sep=comma] {\tablelocaldeltas};
\addplot+[const plot, mark=nonee] table[x=JGroups-nocoord-x, y=JGroups-nocoord-y, col sep=comma] {\tablelocaldeltas};
\legend{EpTO, JGroups-coord, JGroups-nocoord}
\end{axis}
\end{tikzpicture}
	\vspace{-2mm} 
	\caption{Local dissemination stretch with churn}
	\vspace{-2mm}
	\label{fig:local-delta-real-churn}   
\end{figure}
In \autoref{fig:local-delta-real-churn}, \epto as worse outliers than before and so does \jgroups-coord.
\begin{table}[hpt]
	\centering
	\caption{Total events sent during a real trace}
\sisetup{table-format=6.1, separate-uncertainty, table-figures-uncertainty = 6, table-align-uncertainty}
\begin{tabular}{lS}
	\toprule
	Protocol &\\
	\midrule
	\epto & 165844.2(2102)\\
	\jgroups-coord & 166183.0(13681)\\
	\jgroups-nocoord & 166585.8(8249)\\
	\bottomrule
\end{tabular}
    \label{table:total-sent-real-churn}
\end{table}
\autoref{table:total-bandwidth-real-churn} is interesting. We see that on average \jgroups seems to deliver more events than \epto, but its standard deviation is way higher than that of \epto, suggesting \epto to be more stable and consistent.
%\begin{figure}[h]
%	\centering
%	\begin{tikzpicture}
\pgfplotsset{width=\linewidth, height=4.7cm}
\begin{axis}[
ymin=160000,
ymax=170000,
enlarge x limits=0.3,
ybar=0,
/pgf/bar width=3mm,
/pgfplots/area legend,
nodes near coords,
legend style={
	anchor=north west,
	at={(0.6,0.97)},
	cells={anchor=west},
	draw=none,
	fill=none
},
every node near coord/.append style={
	rotate=90,
	anchor=north,
	font=\tiny,
	xshift=5mm,
	yshift=1mm,
},
ymajorgrids,
xtick=data,
xlabel=Protocol,
xticklabel style={text height=1.5ex, rotate=30}, 
scaled ticks=false, tick label style={/pgf/number format/fixed},
ylabel={Events sent},
symbolic x coords={EpTO, JGroups-nocoord, JGroups-coord},
]
% 100-1sec real trace
\addplot[mark=none, pattern=north east lines, error bars/.cd,y dir=both, y explicit] coordinates {
	% Events sent
	(EpTO, 165844.2) +- (0, 210.222976860287)
	(JGroups-nocoord, 166585.8) +- (0, 824.998303028558)
	(JGroups-coord, 166183) +- (0, 1368.95507596122)
};
\end{axis}
\end{tikzpicture}
%	\vspace{-2mm} 
%	\caption{Total events sent per experiment during churn on average}
%	\vspace{-2mm}
%	\label{fig:total-events-real-churn}  
%\end{figure}
\subsection{Problems encountered}
The \epto simulation and theory is assuming we can have balls of infinite size. In practice, packet loss happens. We must therefore limit the ball size to a practical number. We use balls with a maximum size of \SI{1432}{\byte}. We choose this number as the default MTU on Ethernet is \SI{1500}{\byte}. We then subtract the bytes required for the IPv4 and UDP headers.
\par
When running \jgroups we sometimes get failed runs, where either some peers will experience holes or peer will deliver out of order events. This only happens when there is churn happening and when the coordinator dies. We have not investigated these problems further. This points to either \jgroups failing when there is too much churn or a bug present in the SEQUENCER implementation.
