\section{Ordering Algorithms}
\label{sec:ordering}
Distributed systems and centralized systems alike, need to preserve the temporal order of events produced by concurrent processes in the system. When there are separated processes that can only communicate through messages, you cannot easily order these messages.
Therefore we need ordering algorithms to overcome this problem.
\par
We have two types of ordering algorithms \autocite{lamport1978time}: the partial order algorithms and the total order algorithms.
\subsection{Partial Order Algorithms}
Assuming S is partially ordered under $\leq$, then the following statements hold for all a, b and c in S:
\begin{itemize}
	\item Reflexivity: $a \leq a$ for all $a \in S$.
	\item Antisymmetry: $a \leq b$ and $b \leq a$ implies $a=b$ .
	\item Transitivity: $a \leq b$  and $b \leq c$  implies $a \leq c$.
\end{itemize}

\subsection{Total Order Algorithms}
A totally ordered set of events is a partially ordered set which satisfies one additional property:
\begin{itemize}
	\item Totality (trichotomy law): For any $a, b \in S$, either $a \leq b$  or $b \leq a$.
\end{itemize}
\par
In other words, total order is an ordering that defines the exact order of every event in the system. On the other hand, partial ordering only defines the order between certain key events that depend on each other. Partial order can be useful since it is less costly to implement. However, in some cases the order of all events is important. For example, imagine we have multiple databases around the world. We want them to appear as if it is only one database. To achieve this, every operation done a particular database would have to be replicated on all the others, we would then have to make sure that every operation is executed in the same order on every database to end up in the same state. \hm{It would be enlightening to be more precise here and add a large-scale distributed application for which total order is important.}\jt{I came up with a possible total order application. However I am not sure if it is the best use case. Maybe Miguel has a better idea.}
